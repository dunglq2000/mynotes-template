\section*{Chapter 7. Lattices and Cryptography}

7.43. $t = \dfrac{\bm{b}_1 \cdot \bm{b}_2}{\lVert \bm{b}_1 \rVert^2}$ và $\bm{b}_2^* = \bm{b}_2 - t \bm{b}_1$ nên suy ra
\begin{equation*}
	\bm{b}_2^* \cdot \bm{b}_1 = \bm{b}_1 (\bm{b}_2 - t \bm{b}_1) = \bm{b}_1 \cdot \bm{b}_2 - t \lVert \bm{b}_1 \rVert^2 = \bm{b}_1 \cdot \bm{b}_2 - \dfrac{\bm{b}_1 \cdot \bm{b}_2}{\lVert \bm{b}_1 \rVert^2} \cdot \lVert \bm{b}_1 \rVert^2 = 0
\end{equation*}
Do đó $\bm{b}_2^* \perp \bm{b}_1$ và $\bm{b}_2^*$ là hình chiếu của $\bm{b}_2$ lên orthogonal complement của $\bm{b}_1$.

7.44. $\lVert \bm{a} - t \bm{b} \rVert^2 = (\bm{a} - t \bm{b})^2 = \bm{a}^2 - 2t \bm{a} \cdot \bm{b} + t^2 \bm{b}^2 = \lVert \bm{a} \rVert^2 + t^2 \lVert \bm{b} \rVert^2 - 2t \bm{a} \cdot \bm{b} \geqslant 0$ với mọi $t \in \RR$.

Cho $\bm{a} - t \bm{b} = 0$ ta có $t = \dfrac{\bm{a} \cdot \bm{b}}{\lVert \bm{b} \rVert^2}$.

Từ đó ta có $(\bm{a} - t \bm{b}) \cdot \bm{b} = \bm{a} \cdot \bm{b} - t \lVert \bm{b} \rVert^2 = \bm{a} \cdot \bm{b} - \dfrac{\bm{a} \cdot \bm{b}}{\lVert \bm{b} \rVert^2} \cdot \lVert \bm{b} \rVert^2 = 0$.

Vì vậy $\bm{a} - t \bm{b}$ là hình chiếu của $\bm{a}$ lên orthogonal complement của $b$ (tương tự 7.43).
	
7.45. Thuật toán Gauss's lattice reduction.

\begin{algorithm}[ht]
	\caption{Gauss's latice reduction algorithm}
	\begin{algorithmic}
		\While{True}
			\If{$\lVert \bm{v}_2 \rVert < \lVert \bm{v}_1 \rVert$}
				\State swap $\bm{v}_1$ and $\bm{v}_2$	
				\State $m \gets \lfloor \bm{v}_1 \cdot \bm{v}_2 / \lVert \bm{v}_1 \rVert^2 \rceil$
			\EndIf
			\If{$m = 0$}
				\State return $(\bm{v}_1, \bm{v}_2)$
			\EndIf
			\State Replace $\bm{v}_2$ with $\bm{v}_2 - m\bm{v}_1$
		\EndWhile
	\end{algorithmic}
\end{algorithm}
	
$\bm{v}_1 = (14, -47)$, $\bm{v}_2 = (-362, -131)$, 6 steps.

$\bm{v}_1 = (14, -47)$, $\bm{v}_2 = (-362, -131)$, 6 steps.

$\bm{v}_1 = (147, 330)$, $\bm{v}_2 = (690, -207)$, 7 steps.

7.46. Do $W^\perp$ là orthogonal complement của $W$ trong $V$ nên nếu $\bm{z} \in W^\perp$ thì $\bm{z} \cdot \bm{y} = 0$, với mọi $\bm{y} \in W$.

Với hai vector $\bm{z}_1, \bm{z}_2 \in W^\perp$ ta có $\bm{z}_1 \cdot \bm{y} = \bm{z}_2 \cdot \bm{y} = 0$, với mọi $\bm{y} \in W$.

Như vậy $(\bm{z}_1 + \bm{z}_2) \cdot \bm{y} = 0 \Rightarrow \bm{z}_1 + \bm{z}_2 \in W^\perp$.

Ta lại có $\alpha \bm{z}_1 \cdot \bm{y} = \alpha \cdot 0 = 0 \Rightarrow \alpha\bm{z}_1 \in W^\perp$ với mọi $\alpha \in \RR$.

Tới đây ta có hai cách giải.

\textit{Cách 1.} Ta có $W \cup W^\perp = \{\bm{0}\}$. Nếu $\bm{u}$ thuộc cả hai tập $W$ và $W^\perp$ thì $\bm{u} \cdot \bm{u} = 0 \Rightarrow \bm{u} = \bm{0}$.

Ký hiệu $U = W + W^\perp$, ta chứng minh $W = V$.

Ta có thể chọn một cơ sở trực chuẩn (orthonormal basis) trong $U$ và mở rộng nó thành cơ sở trực chuẩn trong $V$. 

Khi đó, nếu $U \neq V$ thì có một phần tử $\bm{e}$ trong cơ sở của $V$ vuông góc với $U$. Do $U$ chứa $W$ và $\bm{e}$ vuông góc với $U$ nên $\bm{e} \in W^\perp$. 

The latter is a subspace of $W$, therefore $e$ is in $W$, which is contrary.

\textit{Cách 2.} Đặt $\{ \bm{e}_1, \bm{e}_2, \cdots, \bm{e}_k \}$ là cơ sở trực chuẩn của không gian con $W$. Với mỗi $\bm{v} \in V$, đặt \[ P(\bm{v}) = \sum_{j=1}^{k} (\bm{v} \cdot \bm{e}_j)  \cdot \bm{e}_j \]

Khi đó với mọi $\bm{v} \in V$ thì $\bm{v} = \underbrace{P(\bm{v})}_{\in W} + \underbrace{(\bm{v} - P(\bm{v}))}_{\in W^\perp}$.

Ở đây $\bm{v} - P(\bm{v}) \in W^\perp$ là vì nếu $j \in \{ 1, 2, \cdots, k \}$ thì 
\begin{align*}
	(\bm{v} - P(\bm{v})) \cdot e_j & = \left(\bm{v} - \sum_{l=1}^{k} (\bm{v} \cdot \bm{e}_l) \cdot \bm{e}_l \right) \cdot \bm{e}_j \\
		& = \bm{v} \cdot \bm{e}_j - \bm{v} \cdot \bm{e}_j = 0	
\end{align*}

Do $\{ \bm{e}_1, \cdots, \bm{e}_k \}$ là cơ sở của $W$, điều này cho ta $\bm{v} - P(\bm{v}) \in W^\perp$.

Như vậy $\lVert \bm{v} \rVert^2 = (a \bm{w} + b \bm{w}')^2 = a^2 \bm{w}^2 + 2ab \bm{w} \bm{w}' + b^2 \bm{w}'^2 = a^2 \lVert \bm{w} \rVert^2 + 0 + b^2 \lVert \bm{w}' \rVert^2 = a^2 \lVert \bm{w} \rVert^2 + b^2 \lvert \bm{w}' \rVert^2$.

