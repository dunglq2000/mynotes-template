\chapter{Abstract Algebra}

Phần này giải các bài tập trong quyển Abstract Algebra: Theory and Applications ở \cite{Judson2012}.

\section*{Chương 3. Groups}

7. Đặt $S = \RR \backslash \{-1\}$ và định nghĩa toán tử 2 ngôi trên $S$ là $a \star b = a + b + ab$. Chứng minh rằng $(S, \star)$ là nhóm Abel

\begin{proof}
    Để chứng minh $(S, \star)$ là nhóm ta chứng minh 3 tiên đề của nhóm.
    \begin{itemize}
        \item Giả sử tồn tại phần tử đơn vị $e$, khi đó $e \star s = s \star e = s$ với mọi $s \in S$. Nghĩa là $e + s + es = s + e + se = s$. Vậy $e + se = 0$ mà $s \neq -1$ nên $e = 0$
        \item Với $e = 0$, giả sử với mọi $s \in S$ có nghịch đảo $s'$. Do $s \star s' = s' \star s = e$ nên $s + s' + ss' = s' + s + s's = e = 0$, tức là $s'(1 + s) = -s$. Vậy $s' = \dfrac{-s}{1 + s}$
        \item Với mọi $a, b, c \in S$, $a \star (b \star c) = a \star (b + c + bc) = a + (b+c+bc) + a (b+c+bc) = a + b + c + ab + bc + ca + abc$ và $(a \star b) \star c = (a + b + ab) \star c = a + b + ab + c + c(a+b+bc) = a + b + c + ab + bc + ca + abc$. Như vậy $a \star (b \star c) = (a \star b) \star c$, tính kết hợp
    \end{itemize}
    Vậy $(S, \star)$ là nhóm.
\end{proof}

39. Gọi $G$ là tập các ma trận $2 \times 2$ với dạng
\[ \begin{pmatrix}
    \cos \theta & -\sin \theta \\ \sin \theta & \cos \theta
\end{pmatrix}\] với $\theta \in \RR$. Chứng minh rằng $G$ là subgroup của $SL_2 (\RR)$.
    
\begin{proof}
        Với $\theta_1, \theta_2 \in \RR$, ta có
        \begin{align*}
        & \begin{pmatrix}
            \cos \theta_1 & -\sin \theta_1 \\ \sin \theta_1 & \cos \theta_1
        \end{pmatrix} \begin{pmatrix}
            \cos \theta_2 & -\sin \theta_2 \\ \sin \theta_2 & \cos \theta_2
        \end{pmatrix} \\
        = & \begin{pmatrix}
            \cos \theta_1 \cos \theta_2 - \sin \theta_1 \sin \theta_2 & -\cos \theta_1 \sin \theta_2 - \sin \theta_1 \cos \theta_2 \\ 
            \sin \theta_1 \cos \theta_2 + \cos \theta_1 \sin \theta_2 & -\sin \theta_1 \sin \theta_2 + \cos \theta_1 \cos \theta_2
        \end{pmatrix} \\
        = & \begin{pmatrix}
            \cos (\theta_1 + \theta_2) & -\sin (\theta_1 + \theta_2) \\
            \sin (\theta_1 + \theta_2) & \cos (\theta_1 + \theta_2)
        \end{pmatrix}
        \end{align*}
        
        Suy ra định thức của tích 2 ma trận là
        \begin{align*}
            \det \Biggl(\begin{pmatrix}
                \cos (\theta_1 + \theta_2) & -\sin (\theta_1 + \theta_2) \\
                \sin (\theta_1 + \theta_2) & \cos (\theta_1 + \theta_2)
            \end{pmatrix}\Biggr)
            = & 1 \cdot 1 = 1
        \end{align*}
        Như vậy phép nhân 2 ma trận có dạng trên đóng trên $SL_2 (\RR)$.
        
        Phần tử đơn vị là $\begin{pmatrix}
        1 & 0 \\ 0 & 1
        \end{pmatrix}$ tương ứng với $\theta = 0$.
        
        Phần tử nghịch đảo là $\begin{pmatrix}
        \cos (-\theta) & -\sin (-\theta) \\ \sin (-\theta) & \cos (-\theta)
        \end{pmatrix}$ suy ra từ công thức định thức ban nãy.
        
        Cuối cùng, phép nhân ma trận có tính kết hợp. Như vậy $G$ là subgroup của $SL_2 (\RR)$.      
\end{proof}

47. Đặt $G$ là nhóm và $g \in G$. Chứng minh rằng \[ Z(G) = \{ x \in G: gx = xg \; \forall \; g \in G \} \] là subgroup của $G$. Subgroup này gọi là \textbf{center} của $G$.

\begin{proof}
    Giả sử trong $G$ có 2 phần tử là $x_1$ và $x_2$ thuộc $Z(G)$. Khi đó \[ x_1 g = g x_1 \text{ và } x_2 g = g x_2 \text{ với mọi } g \in G. \]
    Xét phần tử $x_1 x_2$, ta có \[ (x_1 x_2) g = x_1 (x_2 g) = x_1 (g x_2) = (g x_1) x_2 = g (x_1 x_2) \] với mọi $g \in G$. Do đó $x_1 x_2 \in Z(G)$ nên $Z(G)$ là subgroup.
\end{proof}

49. Cho ví dụ về nhóm vô hạn mà mọi nhóm con không tầm thường của nó đều vô hạn

Ví dụ tập $\ZZ$ và phép cộng số nguyên. Khi đó mọi nhóm con của $\ZZ$ có dạng $n\ZZ$ với $n \in \ZZ$. Ví dụ

$2\ZZ = \{\cdots, -4, -2, 0, 2, 4, \cdots\}$ với phần tử sinh là $2$

$n\ZZ = \{\cdots, -2n, -n, 0, n, 2n, \cdots\}$ với phần tử sinh là $n$

54. Cho $H$ là subgroup của $G$ và \[ C(H) = \{g \in G: gh = hg \; \forall \; h \in H\}. \] Chứng minh rằng $C(H)$ là subgroup của $G$. Subgroup này được gọi là \textbf{centralizer} của $H$ trong $G$.

\begin{proof}
    Gọi $g_1$ và $g_2$ thuộc $C(H)$. Khi đó $g_1 h = h g_1$ và $g_2 h = h g_2$ với mọi $h \in H$

    Xét phần tử $g_1 g_2$, với mọi $h \in H$ ta có \[ (g_1 g_2) h = g_1 (g_2 h) = g_1 (h g_2) = (g_1 h) g_2 = (h g_1) g_2 = h (g_1 g_2). \]
    Như vậy $g_1 g_2 \in C(H)$, từ đó $C(H)$ là subgroup của $G$
\end{proof}

\section*{Chương 5. Permutation Groups}

13. Đặt $\sigma = \sigma_1 \cdots \sigma_m \in S_n$ là tích của các cycle độc lập. Chứng minh rằng order của $\sigma$ là LCM của độ dài các cycle $\sigma_1, \cdots, \sigma_m$.

\begin{proof}
    Đặt $l_i$ là độ dài cycle $\sigma_i$ ($i = 1, \cdots m$). Khi đó $\sigma_i^{k_i l_i}$ sẽ ở dạng các cycle độ dài 1 ($k_i \in \ZZ$).

    Từ đó, $\sigma^l = \sigma_1^l \cdots \sigma_m^l = (1)\cdots(n)$ nếu $l = k_1 l_1 = \cdots k_m l_m$. Số $l$ nhỏ nhất thỏa mãn điều kiện này là $\lcm(l_1, \cdots, l_m)$ (đpcm).
\end{proof}

23. Nếu $\sigma$ là chu trình với độ dài lẻ, chứng minh rằng $\sigma^2$ cũng là chu trình.
\begin{proof}
    Giả sử $\sigma = (g_1, g_2, \cdots, g_{n-1}, g_n)$ với $n$ lẻ. 
    Khi đó \[\sigma^2 = (g_1, g_3, \cdots, g_n, g_2, g_4, \cdots, g_{n-1})\] cũng là chu trình.
\end{proof}

30. Cho $\tau = (a_1, a_2, \cdots, a_k)$ là chu trình độ dài $k$.

\begin{enumerate}
    \item[(a)] Chứng minh rằng với mọi hoán vị $\sigma$ thì \[ \sigma \tau \sigma^{-1} = (\sigma(a_1), \sigma(a_2), \cdots, \sigma(a_k)) \] là chu trình độ dài $k$.
    \item[(b)] Gọi $\mu$ là chu trình độ dài $k$. Chứng minh rằng tồn tại hoán vị $\sigma$ sao cho $\sigma \tau \sigma^{-1} = \mu$
\end{enumerate}

\begin{proof}
    Để chứng minh 2 mệnh đề trên ta cần chú ý một số điều.

    \begin{enumerate}
        \item [(a)] Ta thấy rằng bất kì phần tử nào khác $a_1, a_2, \cdots, a_k$ thì khi qua $\tau$ không đổi, do đó khi đi qua $\sigma \tau \sigma^{-1}$ thì chỉ đi qua $\sigma \sigma^{-1}$ và cũng không đổi. Nói cách khác các phần tử $a_1, a_2, \cdots, a_k$ vẫn nằm trong chu trình nên ta có đpcm.
        \item [(b)] Từ câu (a), với $\mu = (b_1, b_2, \cdots, b_k)$ thì ta chọn $\sigma$ sao cho $b_i = \sigma(a_i)$.
    \end{enumerate}
\end{proof}

\section*{Chương 6. Cosets}

11. Gọi $H$ là subgroup của nhóm $G$ và giả sử $g_1, g_2 \in G$. Chứng minh các mệnh đề sau là tương đương:

\begin{itemize}
    \item[(a)] $g_1 H = g_2 H$
    \item[(b)] $H g_1^{-1} = H g_2^{-1}$
    \item[(c)] $g_1 H \subseteq g_2 H$
    \item[(d)] $g_2 \in g_1 H$
    \item[(e)] $g_1^{-1} g_2 \in H$ 
\end{itemize}

\begin{proof}
    Từ (a) ra (b): Ta đã biết các coset là rời nhau hoặc trùng nhau, do đó với mọi $g_1 h \in g_1 H$, tồn tại $g_2 h' \in g_2 H$ mà $g_1 h = g_2 h'$. Suy ra $(g_1 h)^{-1} = (g_2 h')^{-1}$ hay $h^{-1} g_1^{-1} = h'^{_1} g_2^{-1}$ (đpcm).

    Từ (a) ra (c): Hiển nhiên.

    Từ (a) ra (d): Với mọi $g_1 h \in g_1 H$, tồn tại $g_2 h' \in g_2 H$ sao cho $g_1 h = g_2 h'$, hay $g_2 = g_1 h h'^{-1}$, đặt $h'' = h h'^{-1}$ thì $h'' \in H$ ($H$ là nhóm con) nên $g_1 h'' \in g_1 H$. Suy ra $g_2 \in g_1 H$.

    Từ (a) ra (e): Tương tự, ta có $g_1 h = g_2 h'$, suy ra $h h'^{-1}= g_1^{-1} g_2 \in H$.
\end{proof}


16. Nếu $g h g^{-1} \in H$ với mọi $g \in G$ và $h \in H$, chứng minh rằng right coset trùng với left coset.

\begin{proof}
    Do $g h g^{-1} \in H$ nên tồn tại $h' \in H$ sao cho $g h g^{-1} = h'$. Tương đương $g h = h' g$ với mọi $h \in H$ nên $g H = H g$. Điều này đúng với mọi $g \in G$ nên các right coset trùng left coset.
\end{proof}

17. Giả sử $[G:H]=2$. Chứng minh rằng nếu $a, b$ không thuộc $H$ thì $ab \in H$.

\begin{proof}
    Ta biết rằng 2 coset ứng với 2 phần tử $g_1, g_2$ bất kì là trùng nhau hoặc rời nhau.

    Do đó với $eH = H$, ta suy ra 2 coset của $G$ là $H$ và $G \backslash H$.

    Vì $a, b \not\in H$ nên coset của chúng trùng nhau. Và nghịch đảo của $a$ cũng nằm trong $G \backslash H$ vì nếu nghịch đảo của $a$ nằm trong $H$ thì $a$ cũng phải nằm trong $H$.

    Suy ra $a^{-1} H = b H$. Nghĩa là tồn tại 2 phần tử $h_1, h_2 \in H$ sao cho $a^{-1} h_1 = b h_2$, tương đương $h_1 h_2^{-1} = a b \in H$ (đpcm).
\end{proof}

21. Gọi $G$ là cyclic group với order $n$. Chứng minh rằng có đúng $\phi(n)$ phần tử sinh của $G$.

\begin{proof}
    Gọi $g$ là một phần tử sinh của $G$. Khi đó $g$ sinh ra tất cả phần tử trong $G$, hay nói cách khác các phần tử trong $G$ có dạng $g^i$ với $0 \leq i < n$.

    Như vậy một phần tử $h = g^i$ cũng là phần tử sinh của $G$ khi và chỉ khi $\gcd(i, n) = 1$ và có $\phi(n)$ số $i$ như vậy (đpcm).
\end{proof}

\section*{Chương 9. Isomorphism}

18. Chứng minh rằng subgroup của $\QQ^*$ gồm các phần tử có dạng $2^m 3^n$ với $m, n \in \ZZ$ là internal direct product tới $\ZZ \times \ZZ$

\begin{proof}
    Xét ánh xạ $\varphi: \QQ^* \rightarrow \ZZ \times \ZZ$, $\varphi(2^m 3^n) = (m, n)$.

    Hàm này là well-defined vì với $m$ cố định thì mỗi phần tử $2^m 3^n$ chỉ cho ra một phần tử $(m, n)$. Tương tự với cố định $n$.

    Hàm này là đơn ánh (one-to-one) vì với $m_1 = m_2$ và $n_1 = n_2$ thì $2^{m_1} 3^{n_1} = 2^{m_2} 3^{n_2}$.

    Hàm này cũng là toàn ánh vì với mỗi cặp $(m, n)$ ta đều tính được $2^m 3^n$.

    Vậy hàm $\varphi$ là song ánh.

    Thêm nữa, 
    \begin{align*}
        \varphi(2^{m_1} 3^{n_1} \cdot 2^{m_2} 3^{n_2})& = \varphi(2^{m_1 + m_2} 3^{n_1 + n_2}) \\
        & = (m_1 + m_2, n_1 + n_2) = (m_1, n_1) + (m_2, n_2) \\
        & = \varphi(2^{m_1} 3^{n_1}) \varphi(2^{m_2} 3^{n_2})
    \end{align*}
    Vậy $\varphi$ là homomorphism, và là song ánh nên là isomorphism.
\end{proof}

20. Chứng minh hoặc bác bỏ: mọi nhóm Abel có order chia hết bởi 3 chứa một subgroup có order là 3.

\begin{proof}
    Gọi order của nhóm Abel là $n=3k$, và $g$ là phần tử sinh của nhóm Abel đó. Như vậy $g^n = g^{3k} = e$.

    Nếu ta chọn $h = g^k$ thì $h^3 = e$, khi đó subgroup được sinh bởi $h$ có order 3 (đpcm).
\end{proof}

21. Chứng minh hoặc bác bỏ: mọi nhóm không phải Abel có order chia hết bởi 6 chứa một subgroup có order 6

\begin{proof}
    Với $\mathcal{S}_3$ có order là 6 nhưng không có nhóm con nào order 6 (nhóm con chỉ có order 1, 2 hoặc 3) (bác bỏ).
\end{proof}

22. Gọi $G$ là group với order 20. Nếu $G$ có các subgroup $H$ và $K$ với order 4 và 5 mà $hk=kh$ với mọi $h \in H$ và $k \in K$, chứng minh rằng $G$ là internal direct product của $H$ và $K$.

\begin{proof}
    Ta chứng minh $H \cap K = \{ e \}$. Giả sử tồn tại phần tử $m \in H \cap K$, khi đó do $m \in H$ nên $mk = km$ với mọi $k \in K$. Tuy nhiên $m \in K$ do đó điều này xảy ra khi và chỉ khi $m = e$.

    Như vậy $H \cap K = \{ e \}$.
\end{proof}

