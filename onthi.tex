\chapter{Olympiad}

\section{Ôn thi ngày 20/11/2023}

\subsection*{Toán tử tuyến tính}

Toán tử tuyến tính là một ánh xạ \[ A: \RR^n \to \RR^m \]

Nếu $A$ là một ma trận cỡ $m \times n$ thì đây là một ánh xạ tuyến tính với phép nhân ma trận với vector $A \cdot \bm{x} = \bm{y}$.

Ở đây $\bm{x} \in \RR^n$ và $\bm{y} \in \RR^m$.

\begin{definition}[Hạt nhân]
    \textbf{Hạt nhân} của ánh xạ tuyến tính $A$ là tập hợp nghiệm của hệ thuần nhất và được ký hiệu là $\ker(A)$. Nói cách khác
    \begin{equation}
        \ker(A) = \{ \bm{x} \in \RR^n: \, A \cdot \bm{x} = \bm{0} \}
    \end{equation}
\end{definition}

\begin{definition}[Ảnh]
    \textbf{Ảnh} của ánh xạ tuyến tính $A$ là tập hợp tất cả giá trị có thể của phép nhân ma trận và được ký hiệu là $\Ima(A)$. Nói cách khác
    \begin{equation}
        \Ima(A) = \{ A \cdot \bm{x}: \, \bm{x} \in \RR^n \}
    \end{equation}
\end{definition}

Tính chất đối với ánh xạ $A: \RR^n \to \RR^m$ là $\dim(\ker A) + \dim(\Ima A) = n$.

\subsection*{Trị riêng và vector riêng}

\begin{definition}[Trị riêng, vector riêng]
    Xét hệ phương trình tuyến tính thuần nhất biểu diễn bởi phép nhân \[ A \cdot \bm{x} = \lambda \cdot \bm{x} \]

    Giá trị $\lambda$ khiến phương trình có nghiệm không tầm thường được gọi là \textbf{trị riêng} (eigenvalue) của ánh xạ tuyến tính.

    Vector $\bm{x}$ là cơ sở của không gian vector nghiệm khi đó được gọi là \textbf{vector riêng} (eigenvector) ứng với trị riêng $\lambda$.
\end{definition}

Lưu ý rằng có thể có nhiều vector riêng tương ứng với một trị riêng.

Để tìm trị riêng ta giải phương trình đặc trưng $\det (A - \lambda I) = 0$ và tìm tất cả nghiệm thực $\lambda$ của phương trình.

Sau đó ta thế từng $\lambda$ vào hệ $A \bm{x} = \lambda \bm{x}$ và tìm cơ sở của không gian nghiệm. Các vector trong cơ sở là vector riêng tương ứng với $\lambda$ đó.

Một số tính chất của trị riêng và vector riêng (giả sử rằng đối với ma trận $A$ cỡ $n \times n$ thì phương trình đặc trưng có đầy đủ $n$ nghiệm thực).

\begin{enumerate}
    \item $\tr A = \lambda_1 + \lambda_2 + \ldots + \lambda_n$
    \item $\det A = \lambda_1 \cdot \lambda_2 \cdots \lambda_n$
\end{enumerate}

Tính chất liên quan đến rank và trace:

\begin{enumerate}
    \item $\tr (AB) = \tr (BA)$
    \item $\rank (AB) \leqslant \min(\rank(A), \rank(B))$
\end{enumerate}

\subsection*{Bài tập}

\textbf{Bài 1}. Cho vector cột $\bm{v} \in \RR^n$. Đặt $A = \bm{v} \cdot \bm{v}^T$. Tìm $\spa A$.

Các cột của $A$ có dạng $\bm{v} \cdot v_1$, $\bm{v} \cdot v_2$, ..., $\bm{v} \cdot v_n$. Như vậy các cột đều tỉ lệ với cột đầu nên $\rank A = 1$.

Suy ra $\dim \ker A = n-1$ và do đó $\lambda = 0$ là nghiệm bậc $n-1$ trong phương trình đặc trưng.

Như vậy phương trình đặc trưng còn một nghiệm $\lambda \neq 0$.

Do $(\bm{v} \cdot \bm{v}^T) \bm{x} = \lambda \bm{x} \Leftrightarrow \bm{v} (\bm{v}^T \cdot \bm{x}) = \lambda \bm{x}$.

Đặt $\bm{v}^T \cdot \bm{x} = \alpha$ thì $\alpha \bm{v} = \lambda \bm{x}$. Suy ra $\bm{x} = \bm{v}$ và do đó $\alpha = \lambda = \lVert \bm{v} \rVert^2$.

Vậy $\spa A = \{ \lVert \bm{v} \rVert^2, 0, 0, \ldots, 0\}$.

\hfill

\textbf{Bài 3}. Cho ma trận $A_{3 \times 3}$. Biết rằng $\tr A = \tr A^{-1} = 0$ và $\det A = 1$. Chứng minh rằng $A^3 = I$.

Phương trình đặc trưng có dạng $P_3(\lambda) = -\lambda^3 + a_2 \lambda^2 + a_1 \lambda + a_0$.

Theo tính chất trên thì $a_2 = \sum \lambda = \tr A = 0$.

Do $\lambda$ là trị riêng nên $A \bm{x} = \lambda \bm{x}$. Do $A$ khả nghịch nên $\dfrac{1}{\lambda} \bm{x} = A^{-1} \bm{x}$.

Nghĩa là $\dfrac{1}{\lambda}$ là trị riêng của ma trận $A^{-1}$. Suy ra $\dfrac{1}{\lambda_1} + \dfrac{1}{\lambda_2} + \dfrac{1}{\lambda_3} = \tr A^{-1} = 0$.

Từ đó suy ra $\lambda_1 \lambda_2 + \lambda_2 \lambda_3 + \lambda_3 \lambda_1 = 0$.

Cuối cùng $\det A = \lambda_1 \cdot \lambda_2 \cdot \lambda_3 = 1$.

Vậy phương trình đặc trưng là $P_3(\lambda) = -\lambda^3 + 1$. Theo định lý Cayley-Hamilton thì $P_3(A) = -A^3 + I = 0$, hay $A^3 = I$.

\hfill

\textbf{Bài 4}. Cho ma trận $A_{n \times n}$, $A_{ij} \geqslant 0$. Giả sử ma trận có đủ $n$ trị riêng thực. Chứng minh rằng $\lambda_1^k + \lambda_2^k + \ldots + \lambda_n^k \geqslant 0$ với mọi $k \in \NN$.

Ta thấy rằng với $k=1$ thì $\lambda_1 + \ldots + \lambda_n = \tr(A) \geqslant 0$.

Vì $\lambda_i$ là thỏa phương trình $A \bm{x} = \lambda_i \bm{x}$ nên nhân hai vế cho $A$ ta có $A \cdot A \bm{x} = A \cdot \lambda_i \bm{x}$. Tương đương với $A^2 \bm{x} = \lambda_i (A \bm{x}) = \lambda_i^2 \bm{x}$.

Nói cách khác, $\lambda_i^2$ là trị riêng của ma trận $A^2$. Thực hiện tương tự ta có $\lambda_i^k$ là trị riêng của ma trận $A^k$.

Do đó $\lambda_1^k + \ldots + \lambda_n^k = \tr(A^k) \geqslant 0$.

\hfill

\textbf{Bài 5}. Cho ma trận $A$ khả nghịch. $X$ là ma trận sao cho $AX + XA = 0$. Chứng minh rằng $\tr X = 0$.

Nhân bên trái hai vế cho $A^{-1}$ ta có $X + A^{-1} X A = 0$. Ta biết rằng $A^{-1} X A$ là ma trận tương đương ma trận $X$ nên $\tr (A^{-1} X A) = \tr X$.

Suy ra $\tr X + \tr X = \tr 0 = 0$. Từ đây có $\tr X = 0$.

\section{RUDN Olympiad 2023}

Lần đầu tiên mình được tham dự thi toán đồng đội theo hình thức MathBoy (trận chiến toán).

Trong cách thi này, mỗi đội có 3 vị trí: người thuyết trình (докладчик), người phản biện (оппонент) và người giám sát (наблюдатель).

Ở mỗi vòng sẽ có 3 đội thi với nhau. Mỗi đội sẽ có 1 vị trí tương ứng với 3 vị trí trên. Sau đây là ví dụ

\begin{table}[ht]
    \centering
    \begin{tabular}{|c|c|c|c|}
        \hline
        & Đội 1 & Đội 2 & Đội 3 \\ \hline
        Vòng 1 & О & Д & Н \\ \hline
        Vòng 2 & Н & О & Д \\ \hline
        Vòng 3 & Д & Н & О \\ \hline
    \end{tabular}
\end{table}

Ở mỗi vòng, đội đóng vai trò người thuyết trình lên bảng ghi bài giải trong thời gian cho phép và thuyết trình về bài giải của đội mình. Đội phản biện có nhiệm vụ phản biện bài thuyết trình đó. Đội giám sát, dựa trên bài thuyết trình cũng như phản biện mà ghi chép lại các lỗi, chỗ khó hiểu, ... và trình lên cho giám khảo.

Ngoài ra, đội thuyết trình trước đó phải trình bài giải viết tay cho giám khảo chấm trước khi lên thuyết trình.

Ở đây có rất nhiều câu chuyện hack não đã xảy ra. Lúc mình thi vòng 1, câu hỏi quá khó nên đội thuyết trình chỉ viết được một ít. Đồng nghĩa việc đội phản biện cũng như đội giám sát ... thất nghiệp, không có gì để nói.

Đối với vòng 2, trận chiến cân bằng hơn, đội mình làm việc giám sát. Dựa trên bài giải của đội thuyết trình, chúng mình thấy những trường hợp chưa được xét tới và có thể bị sai, do đó cả ba đội đều có điểm (đội thuyết trình có nhiều điểm nhất vì các bạn giải hơn 1 nửa rồi).

Đối với vòng 3, đội mình thuyết trình. Đội mình clear bài đó nên giành điểm tuyệt đối cho phần thuyết trình. Tuy nhiên các bạn phản biện cũng không vừa, vẫn cố gắng bắt một số lỗi do trình bày quá cô đọng. Kết quả là đội mình (thuyết trình) full điểm cho vòng 3, đội phản biện được 3 điểm.
