\chapter{Lattice-based crypto}

\section{Introduction}

\begin{definition}[Lattice]
    Xét các vector thuộc $\RR^n$ độc lập tuyến tính $\{ \bm{v}_1, \ldots, \bm{v}_d \}$. \textbf{Lattice} là tập
    \begin{equation}
        L = \{ a_1 \bm{v}_1 + \ldots + a_d \bm{v}_d : a_i \in \ZZ \}
    \end{equation}
\end{definition}

Tương tự với định nghĩa không gian vector, một \textbf{tập sinh} (hay \textbf{basis}) là bất cứ tập hợp các vector độc lập tuyến tính mà sinh ra $L$.

Hai tập sinh luôn có cùng số phần tử. Khi đó, số vector trong tập sinh được gọi là \textbf{số chiều} (hay \textbf{dimension}).

Giả sử $\{ \bm{v}_1, \bm{v}_2, \ldots, \bm{v}_d \}$ là một cơ sở của $L$. Tương tự, $\{ \bm{w}_1, \bm{w}_2, \ldots, \bm{w}_d \}$ là một cơ sở khác của $L$.

Ta có thể viết mỗi $\bm{w}_i$ là tổ hợp tuyến tính của các vector $\bm{v}$ như sau
\begin{align*}
    \bm{w}_1 & = a_{11} \bm{v}_1 + a_{12} \bm{v}_2 + \ldots + a_{1d} \bm{v}_d \\
    \bm{w}_2 & = a_{21} \bm{v}_1 + a_{22} \bm{v}_2 + \ldots + a_{2d} \bm{v}_d \\
    \vdots & \\
    \bm{w}_d & = a_{d1} \bm{v}_1 + a_{d2} \bm{v}_2 + \ldots + a_{dd} \bm{v}_d
\end{align*}

Khi đó, nếu viết các vector $\bm{w}_i$ thành hàng của ma trận $\bm{W}$ và $\bm{v}_j$ thành hàng của ma trận $\bm{V}$ thì biểu diễn trên tương đương với
\begin{equation*}
    \bm{W} = \begin{pmatrix}
        a_{11} & a_{12} & \cdots & a_{1d} \\
        a_{21} & a_{22} & \cdots & a_{2d} \\
        \vdots & \vdots & \ddots & \vdots \\
        a_{d1} & a_{d2} & \cdots & a_{dd}
    \end{pmatrix} \cdot \bm{V}
\end{equation*}

Đặt
\begin{equation*}
    \bm{A} = \begin{pmatrix}
        a_{11} & a_{12} & \cdots & a_{1d} \\
        a_{21} & a_{22} & \cdots & a_{2d} \\
        \vdots & \vdots & \ddots & \vdots \\
        a_{d1} & a_{d2} & \cdots & a_{dd}
    \end{pmatrix}
\end{equation*}

Do $\bm{W}$ và $\bm{V}$ là các cơ sở của $L$ nên nếu các vector $\bm{w}_i$ có thể biểu diễn qua các vector $\bm{v}_j$ thì ngược lại, các vector $\bm{v}_j$ cũng có thể được biểu diễn qua các vector $\bm{w}_i$.

Suy ra ma trận $\bm{A}$ là ma trận khả nghịch. Do $a_{ij} \in \ZZ$ theo định nghĩa lattice, định thức của $\bm{A} \in \ZZ$.

Hơn nữa, vì
\begin{equation*}
    I = \bm{A} \cdot \bm{A}^{-1} \Rightarrow 1 = \det (\bm{A}) \cdot \det (\bm{A}^{-1})
\end{equation*}
nên $\det(\bm{A}) = \pm 1$.

\begin{definition}[Fundamental domain]
    Cho lattice $L$ có số chiều là $d$ với cơ sở gồm các vector $\{ \bm{v}_1, \bm{v}_2, \ldots, \bm{v}_d \}$. Ta gọi \textbf{fundamental domain} (hay \textbf{fundamental parallelepiped}) của $L$ ứng với cơ sở trên là tập
    \begin{equation}
        \mathcal{F} (\bm{v}_1, \ldots, \bm{v}_d) = \{ t_1 \bm{v}_1 + \ldots + t_d \bm{v}_d : 0 \leqslant t_i < 1 \}
    \end{equation}
\end{definition}

Trong mặt phẳng $Oxy$ với hai vector $\bm{u}$ và $\bm{v}$ không cùng phương thì fundamental domain là hình bình hành tạo bởi hai vector đó.

\begin{remark}
    Gọi $L \subset \RR^n$ là lattice với số chiều là $n$ và gọi $\mathcal{F}$ là fundamental domain của $L$. Khi đó mọi vetor $\bm{w} \in \RR^n$ đều có thể viết dưới dạng
    \begin{equation*}
        \bm{w} = \bm{t} + \bm{v}
    \end{equation*}
    với $\bm{t}$ duy nhất thuộc $\mathcal{F}$ và $\bm{v}$ duy nhất thuộc $L$.

    Một cách tương đương, hợp của các fundamental domains
    \begin{equation*}
        \mathcal{F} + \bm{v} = \{ \bm{t} + \bm{v} : \bm{t} \in \mathcal{F} \}
    \end{equation*}
    với $\bm{v}$ là các vector trong $L$, sẽ cover hết $\RR^n$.
\end{remark}

\begin{proof}
    Để chứng minh nhận xét trên, giả sử $\{ \bm{v}_i : 1 \leqslant i \leqslant n \}$ là cơ sở của $L$. Khi đó các $\bm{v}_i$ độc lập tuyến tính nên cũng là cơ sở của $\RR^n$.

    Ta viết các vector $\bm{w} \in \RR^n$ dưới dạng tổ hợp tuyến tính của $\bm{v}_i$ và tách hệ số trước mỗi vector thành phần nguyên và phần lẻ. Phần nguyên cho vector trong $L$ và phần lẻ cho vector trong $\mathcal{F}$.

    Để chứng minh tính duy nhất của tổ hợp, xét hai cách biểu diễn khác nhau của $\bm{w}$ và chứng minh hai cách đó là một.
\end{proof}

\begin{theorem}[Bất đẳng thức Hadamard]
    Cho lattice $L$, lấy cơ sở bất kỳ của $L$ là các vector $\bm{v}_1$, ..., $\bm{v}_n$ và gọi $\mathcal{F}$ là fundamental domain cho $L$. Khi đó
    \begin{equation}
        \det L = \text{Vol} (\mathcal{F}) \leqslant \lVert \bm{v}_1 \rVert \cdot \lVert \bm{v}_2 \rVert \cdots \lVert \bm{v}_n \rVert
    \end{equation}
\end{theorem}

Cơ sở càng gần với trực giao thì bất đẳng thức Hadamard trên càng trở thành đẳng thức.

\subsection*{Short vectors trong lattice}

Các bài toán liên quan tới lattice mà chúng ta cần quan tâm để xây dựng các thuật toán mã hóa lattice-based.

\begin{enumerate}
    \item \textbf{Shortest Vector Problem} (\texttt{SVP}): Tìm vector khác không có độ dài ngắn nhất trong lattice $L$, nghĩa là tìm vector $\bm{v} \in L$ mà $\lVert v \rVert$ nhỏ nhất;
    \item \textbf{Closest Vector Problem} (\texttt{CVP}): Cho trước vector $\bm{w} \in \RR^n$ mà không nằm trong $L$, tìm vector $\bm{v} \in L$ gần với $\bm{w}$ nhất, nghĩa là tìm $\bm{v} \in L$ sao cho $\lVert \bm{w} - \bm{v} \rVert$ nhỏ nhất.
\end{enumerate}

\subsection*{Thuật toán Babai}

Thuật toán này giúp tìm một cơ sở "đủ tốt" để giải \texttt{apprCVP}.

\begin{theorem}[Thuật toán Babai tìm Closest Vertex]
    Gọi $L \subset \RR^n$ là lattice với cơ sở là $\bm{v}_1$, $\bm{v}_2$, \dots, $\bm{v}_n$ và gọi $\bm{w} \in \RR^n$ là vector bất kỳ. 
    
    Nếu các vector trong cơ sở trực giao nhau thì thuật toán Babai sẽ giải được \texttt{CVP}.

    Nếu các vector trong cơ sở gần như trực giao thì thuật toán Babai sẽ giải được \texttt{apprCVP}.

    Ngược lại, nếu các vector trong cơ sở không trực giao (nhiều) thì kết quả thuật toán trả về sẽ xa hơn vector gần với $\bm{w}$.
\end{theorem}

\begin{algorithm}[htb]
    \caption{Babai's Closest Vertex Algorithm}
    \begin{algorithmic}
        \State Biểu diễn $\bm{w} = t_1 \bm{v}_1 + t_2 \bm{v}_2 + \ldots + t_n \bm{v}_n$ với $t_1, \ldots, t_n \in \RR$.
        \State Đặt $a_i = \lfloor t_i \rceil$ với $i = 1, \ldots, n$.
        \State Trả về vector $\bm{v} = a_1 \bm{v}_1 + a_2 \bm{v}_2 + \ldots + a_n \bm{v}_n$.
    \end{algorithmic}
\end{algorithm}

Mình có thể thấy rằng thuật toán Babai làm tròn hệ số để đưa trả về vector gần với $\bm{w}$.

\section{Lattice reduction}

Mục tiêu của việc giải các bài toán \texttt{SVP} hay \texttt{CVP} (hay các biến thể của chúng) là tìm vector ngắn trong lattice.

Đối với lattice 2 chiều chúng ta có thuật toán Gauss reduction rất hữu dụng.

\subsection*{Gaussian lattice reduction}

Giả sử ta có $L \subset \RR^2$ là lattice 2 chiều với cơ sở là $\bm{v}_1$ và $\bm{v}_2$.

Ta có thể đổi chỗ $\bm{v}_1$ và $\bm{v}_2$ nếu cần thiết để luôn đảm bảo $\lVert \bm{v}_1 \rVert < \lVert \bm{v}_2 \rVert$.

Để làm nhỏ $\bm{v}_2$ ta có thể trừ đi một bội số của $\bm{v}_1$. Nếu ta sử dụng cách giống như thuật toán trực giao Gram-Schmidt thì
\begin{equation*}
    \bm{v}_2^* = \bm{v}_2 - \dfrac{\bm{v}_1 \cdot \bm{v}_2}{\lVert \bm{v}_1 \rVert^2} \bm{v}_1
\end{equation*}

Khi đó $\bm{v}_2^*$ sẽ trực giao với $\bm{v}_1$. Tuy nhiên vector này không nằm trong $L$ vì hệ số $\dfrac{\bm{v}_1 \cdot \bm{v}_2}{\lVert \bm{v}_1 \rVert^2}$ có thể không phải số nguyên. Để khắc phục ta chỉ việc lấy phần nguyên của hệ số này. Nghĩa là ta thay $\bm{v}_2$ bằng vector
\begin{equation*}
    \bm{v}_2 - m \bm{v}_1 \quad \text{với}\ m = \left\lfloor \dfrac{\bm{v}_1 \cdot \bm{v}_2}{\lVert \bm{v}_1 \rVert^2} \right\rceil
\end{equation*}

Nếu $\bm{v}_2$ vẫn lớn hơn $\bm{v}_1$ thì ta dừng thuật toán. Ngược lại ta đảo vị trí $\bm{v}_1$ và $\bm{v}_2$ rồi lặp lại bước trên. Thuật toán như sau

\begin{algorithm}[htb]
    \caption{Gaussian lattice reduction}
    \begin{algorithmic}  
        \Require Lattice 2 chiều $L \subset \RR^2$ với cơ sở $\bm{v}_1$ và $\bm{v}_2$
        \Ensure Cơ sở mới $\bm{v}_1^*$ và $\bm{v}_2^*$ gần như trực giao nhau
        \State Nếu $\lVert \bm{v}_2 \rVert < \lVert \bm{v}_1 \rVert$ thì ta đổi chỗ $\bm{v}_1$ và $\bm{v}_2$.
        \State Tính $m = \left\lfloor \bm{v}_1 \cdot \bm{v}_2 / \lVert \bm{v}_1 \rVert^2 \right\rceil$.
        \State Nếu $m = 0$ thì trả về cơ sở gồm $\bm{v}_1$ và $\bm{v}_2$.
        \State Thay $\bm{v}_2$ bởi $\bm{v}_2 - m \bm{v}_1$.
    \end{algorithmic}
\end{algorithm}

\subsection*{Thuật toán LLL}

Khi số chiều lattice lớn hơn 2, chúng ta sử dụng thuật toán LLL.

\section{Phương pháp Coppersmith}

Phương pháp Coppersmith được dùng để tìm nghiệm nhỏ của đa thức trên modulo. Phần này tham khảo chính từ quyển \cite{Galbraith}.

\subsection*{Ý tưởng}

Giả sử ta có phương trình $F(x) \equiv 0 \bmod M$​. Với số $X$​ cố định cho trước, phương pháp Coppersmith cho phép tìm nghiệm $x_0$ nhỏ thỏa mãn $\lvert x_0 \rvert \leqslant X$​.

Ý tưởng của phương pháp này là thay vì tìm nghiệm $x_0$ của $F(x)$​ trên modulo $M$​, chúng ta sẽ mở rộng lên, tìm một hàm $G(x)$​ nào đó mà có nghiệm $x_0$​ trên $\mathbb{Z}$​.

Đơn giản nhất là $G(x) = k \cdot F(x) + M \cdot g(x)$​, $k \in \mathbb{Z}$ và $\deg g(x) \leqslant \deg F(x)$. Rõ ràng khi modulo hai vế cho $M$​ thì $G(x_0) = F(x_0) = 0 \bmod M$.

Phương pháp này giúp tìm nghiệm của một đa thức bậc nhỏ modulo $M$​. Do đó giả sử đặt:
\begin{equation*}
    F(x) = a_0 + a_1 x + \cdots + a_d x^d
\end{equation*}​
với $a_i \in \mathbb{Z}$.

Lúc này chúng ta sẽ tìm hàm $G(x)$​ trên với hệ số nhỏ.

Giả sử $g(x) = b_0 + b_1 x + \cdots + b_d x^d$ với $b_i \in \mathbb{Z}$.

Khi đó $G(x) = (k \cdot a_0 + M \cdot b_0) + (k \cdot a_1 + M \cdot b_1) x + \cdots + (k \cdot a_d + M \cdot b_d) x^d$. 

Ta mong muốn các hệ số $k \cdot a_0 + M \cdot b_0$​, $k \cdot a_1 + M \cdot b_1$​, $\cdots$​, $k \cdot a_d + M \cdot b_d$ nhỏ so với $M$​.

Do đó với số $X$​ cho trước, nếu ta xây lattice ($d+2$ vector) sau:
\begin{equation*}
    \begin{array}{cccccccc}
        \bm{v}_0 & \Leftarrow & M & 0 & 0 & \cdots & 0 & 0 \\
        \bm{v}_1 & \Leftarrow & 0 & MX & 0 & \cdots & 0 & 0 \\
        \vdots & \vdots & \vdots & \vdots & \vdots & \ddots & \vdots & \vdots \\
        \bm{v}_d & \Leftarrow & 0 & 0 & 0 & \cdots & 0 & MX^d \\
        \bm{v}_{d+1} & \Leftarrow & a_0 & a_1 X & a_2 X^2 & \cdots & a_{d-1} X^{d-1} & a_d X^d
    \end{array}
\end{equation*}

Khi đó hệ số của mỗi dòng từ $\bm{v}_0$​ tới $\bm{v}_d$​ là $b_0$​ tới $b_d$​. Còn hệ số của $\bm{v}_{d+1}$​ là $k$​.

Tuy nhiên chúng ta thường sẽ biến đổi để đa thức trở thành monic ($a_d = 1$​). Khi đó chúng ta bỏ đi $v_d$​ và còn $d+1$​ vector trong lattice.

\subsection*{Cải tiến thuật toán}

\subsubsection*{Dạng đơn giản}

Giả sử $k(x)$​ có bậc là $h$​. Đặt $k(x) = c_0 + c_1 x + \cdots c_h x^h$​.

Khi đó ​
\begin{align*}
    G(x) = & k(x) \cdot F(x) + M \cdot g(x) \\
        = & (c_0 + c_1 x + \cdots c_h x^h) \cdot (a_0 + a_1 x + \cdots \\
        + & x^d) + M \cdot (b_0 + b_1 x + \cdots + b_d x^d) \\
        = & c_0 \cdot F(x) + c_1 \cdot xF(x) + \cdots c_h \cdot x^h F(x) + M \cdot (b_0 + b_1 x + \cdots b_d x^d)
\end{align*}

Lúc này mỗi đại lượng $F(x)$​, $xF(x)$​, $\cdots$​, $x^h F(x)$​ sẽ khiến hệ số của $F(x)$​ ban đầu tăng bậc. Nói cách khác hệ số $a_i$ của $x^i$​ trong $F(x)$​ sẽ là hệ số của $x^{i+j}$​ trong $x^j F(x)$​.

Sách giáo khoa nói rằng nếu mình chọn $h=d-1$​ thì phương pháp Coppersmith sẽ cho ra kết quả nếu $X$​ được chọn thỏa $X \approx M^{1/(2d-1)}$.

Tương tự, mình sẽ có các vector trong lattice như sau:
\begin{equation*}
    \begin{array}{ccccccccccc}
        \bm{v}_0 & \Leftarrow & M & 0 & 0 & \cdots & 0 & 0 & 0 & 0 & \cdots \\​
        \bm{v}_1 & \Leftarrow & 0 & MX & 0 & \cdots & 0 & 0 & 0 & 0 & \cdots \\
        \vdots & \vdots & \vdots & \vdots & \vdots & \vdots & \vdots & \vdots & \vdots & \vdots & \vdots \\
        \bm{v}_{d-1} & \Leftarrow & 0 & 0 & 0 & \cdots & MX^{d-1} & 0 & 0 & 0 & \cdots \\
        \bm{v}_d & \Leftarrow & a_0 & a_1 X & a_2 X^2 & \cdots & a_{d-1} X^{d-1} & X^d & 0 & 0 & \cdots
    \end{array}
\end{equation*}

Sau đó thêm các vector khi shift vào (tương ứng với $xF(x)$​, $x^2 F(x)$​, $\cdots$).​
\begin{equation*}
    \begin{array}{cccccccccccc}
        \bm{v}_{d+1} & \Leftarrow & 0 & a_0 X & a_1 X^2 & a_2 X^3 & \cdots & a_{d-1} X^d & a_d X^{d+1} & 0 & 0 & \cdots \\
        \bm{v}_{d+2} & \Leftarrow & 0 & 0 & a_0 X^2 & a_1 X^3 & \cdots & a_{d-2} X^d & a_{d-1} X^{d+1} & a_d X^{d+2} & 0 & \cdots
    \end{array}
\end{equation*}

Tuy nhiên vấn đề ở đây là bound của nghiệm bị thu hẹp lại. Ban nãy mình nói rằng nếu chọn $h=d-1$​ thì nghiệm cho kết quả nếu $X \approx M^{1/(2d-1)}$​. Ở đây $M = 10001$​ nên $X \approx 4$​. Trong khi ở bài viết trước thì $X = 10$​. Do đó có thể thấy việc mở rộng này đôi khi kém hiệu quả tùy thuộc vào $h$​.

\subsubsection*{Dạng nâng cao}

Coppersmith đã đề xuất ý tưởng như sau:

\begin{lemma}[Coppersmith]
    Với $0 < \epsilon < \min \{0,18; 1/d\}$. Đặt $F(x)$​ là đa thức monic bậc $d$​ có một hoặc nhiều nghiệm $x_0$​ trên modulo $M$​ sao cho $\lvert x_0 \rvert \leqslant \frac{1}{2} M^{1/d-\epsilon}$. Khi đó $x_0$​ có thể tính với thời gian đa thức giới hạn bởi $d$​, $1/\epsilon$​ và $\log(M)$.
\end{lemma}

Để chứng minh bổ đề này thì Coppersmith đã xây dựng một hệ lattice để tính toán và cũng là cách xây dựng lattice sẽ đề cập tới đây.

Chúng ta biết rằng $x_0$​ là nghiệm của đa thức $F(x) \equiv 0 \bmod M$​. Do đó ta suy ra $F(x_0)^k \equiv 0 \bmod M^k$​.

Từ nhận xét này, mình mở rộng phần cơ bản lên tìm nghiệm trong modulo $M^{h-1}$​ với $h$ là một số được chọn trước.

Với $k(x) = c_0 + c_1 x + \cdots + c_{d-1} x^{d-1}$ biểu diễn việc shift thành $F(x)$, $xF(x)$​, $x^2 F(x)$, ..., $x^{d-1} F(x)$​.

Ta xét $h$​ đa thức sau:
\begin{equation*}
    \begin{array}{ccc}
        M^{h-1} F(x)^0 k(x) & \equiv & 0 \bmod M^{h-1} \\
        M^{h-2} F(x)^1 k(x) & \equiv & 0 \bmod M^{h-1} \\
        \cdots & \ddots & \cdots \\
        M^0 F(x)^{h-1} k(x) & \equiv & 0 \bmod M^{h-1}
    \end{array}
\end{equation*}

Với mỗi đa thức $M^{h-1-j} F(x)^j$​ chúng ta có $d$​ lần shift tương ứng với từng hệ số của $k(x)$​. Cụ thể là $M^{h-1-j}F(x)^j$​, $M^{h-1-j}F(x)^j x$​, $M^{h-1-j}F(x)^j x^2$​, ..., $M^{h-1-j}F(x)^j x^{d-1}$​.

Do đó có tất cả $dh$​ vector trong lattice. Số $h$​ thường được chọn sao cho $dh \approx \epsilon$. Và chặn nghiệm $X$​ có thể chọn là $\frac{1}{2}M^{1/d - \epsilon}$​ theo như bổ đề.

Như vậy mình xây lattice như sau:

\textbf{Bước 1.} Với $M^{h-1} F(x)^0$​ thì các vector sau lần lượt tương ứng với
\begin{equation*}
    M^{h-1} F(x)^0, M^{h-1} F(x)^0 x, \ldots, M^{h-1} F(x)^0 x^{d-1}.
\end{equation*}

Nói cách khác thì
\begin{equation*}
    \begin{array}{ccccccccc}
    \bm{v}_0 & \Leftarrow & M^{h-1} & 0 & 0 & \cdots & 0 & 0 & \cdots \\
    \bm{v}_1 & \Leftarrow & 0 & M^{h-1} X & 0 & \cdots & 0 & 0 & \cdots \\
    \vdots & \vdots & \vdots & \vdots & \vdots & \ddots & \vdots & \vdots & \vdots \\
    \bm{v}_{d-1} & \Leftarrow & 0 & 0 & 0 & \cdots & M^{h-1} X^{d-1} & 0 & \cdots
    \end{array}
\end{equation*}

\textbf{Bước 2.} Với $M^{h-2} F(x)$​ thì các vector sau lần lượt tương ứng với
\begin{equation*}
    M^{h-2} F(x)^1, M^{h-2} F(x)^1 x, \ldots, M^{h-2} F(x)^1 x^{d-1}
\end{equation*}

Nói cách khác thì
\begin{equation*}
    \begin{array}{cccccccccc}
        \bm{v}_d & \Leftarrow & Ma_0 & M a_1 X & M a_2 X^2 & \cdots & M a_{d-1} X^{d-1} & M X^d & \cdots & \cdots \\ 
        \bm{v}_{d+1} & \Leftarrow & 0 & Ma_0 X & Ma_1 X^2 & \cdots & M a_{d-2} X^{d-1} & M a_{d-1} X^d & M X^{d+1} & \cdots \\
        \vdots & \vdots & \vdots & \vdots & \vdots & \vdots & \vdots & \vdots & \vdots & \vdots \\
        \bm{v}_{2d-1} & \Leftarrow & 0 & 0 & 0 & \cdots & M a_0 X^{d-1} & M a_1 X^d & \cdots & \cdots
    \end{array}
\end{equation*}

\textbf{Bước thứ $j$}. Cứ như vậy với $M^{h-1-j}F(x)^j$​.

Chạy LLL trên lattice trên sẽ cho kết quả \emoji{zany-face}.

\section{Các thuật toán mã hóa dựa trên lattice}

\subsection*{Congruential public key cryptosystem}

Phần này tham khảo trong \cite{Hoffstein2014}.

Trong thuật toán này, ta chọn số nguyên tố $q$ làm public parameter.

Sau đó chọn hai số $f$ và $g$ làm secret key. Hai số này phải thỏa mãn các điều kiện \[ f < \sqrt{q/2}, \quad \sqrt{q/4} < g < \sqrt{q/2}, \quad \gcd(f, qg) = 1 \]

Tính $h = f^{-1} g \pmod q$. Khi đó public key là $h$.

\textbf{Encryption}. Để encrypt message $m$ với số random $r$ thỏa mãn \[ 0 < m < \sqrt{q/4}, \quad 0 < r < \sqrt{q/2} \]

Ta tính $e = rh + m \pmod q$ là ciphertext với $0 < e < q$.

\textbf{Decryption}. Để decrypt ciphertext $e$ ta tính \[ a = fe \pmod q, \quad b = f^{-1} a \pmod g \]

Lưu ý $f^{-1}$ là nghịch đảo modulo $g$. Khi đó $b \equiv m$ là message ban đầu.

\begin{proof}
    Để chứng minh rằng số $b$ sau khi tính toán bằng chính xác $m$ ban đầu ta cần xem xét điều kiện của secret key và public key.

    Đầu tiên ta có \[ a \equiv fe \equiv f(rh + m) = f(r f^{-1} g + m) = rg + fm \pmod q \]

    Từ điều kiện của $f$, $g$, $r$ và $m$ ta có \[ rg + fm < \sqrt{\dfrac{q}{2}} \cdot \sqrt{\dfrac{q}{2}} + \sqrt{\dfrac{q}{2}} \cdot \sqrt{\dfrac{q}{4}} < q \]

    Nói cách khác $rg + fm$ giữ nguyên giá trị trong phép modulo $q$, hay $a \equiv rg + fm$.

    Suy ra $b = f^{-1} a = f^{-1} (rg + fm) = m \pmod g$ (giá trị $a$ không thay đổi khi chuyển từ modulo $q$ sang modulo $g$). Do $0 < m < \sqrt{q/4}$ và $\sqrt{q/4} < g < \sqrt{q/2}$ nên $m < g$. Nói cách khác $b$ bằng đúng $m$ ban đầu.
\end{proof}

Để tấn công hệ mật mã này ta xây dựng lattice. Để ý rằng $h = f^{-1} g \pmod q$, hay $fh + kq = g$ với $k \in \ZZ$.

Ta thấy rằng $f \cdot (h, 1) + k \cdot (q, 0) = (g, f)$. Như vậy lattice gồm hai vector $(h, 1)$ và $(q, 0)$. Thuật toán tối giản Gauss sẽ hoạt động trong trường hợp này (số chiều bằng 2).
