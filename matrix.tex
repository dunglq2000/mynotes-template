\chapter{Ma trận}

Trong các bài viết của về đại số tuyến tính:

\begin{itemize}
    \item Vector sẽ được ký hiệu bởi chữ thường in đậm (ví dụ $\bm{v}, \bm{x}, \ldots$)​; 
    \item Ma trận sẽ được ký hiệu bởi chữ hoa in đậm (ví dụ $\bm{A},\ \bm{B}, \ldots$​);
    \item Các đại lượng vô hướng (số) được ký hiệu bởi chữ thường không in đậm (ví dụ $x_1,\ N,\ t, \ldots$).
\end{itemize}

\section{Định thức và hạng ma trận}

\subsection*{Định thức ma trận}

\begin{definition}[Nghịch thế]
    Cho tập hợp $A = \{1, 2, \cdots, n\}$ và xét hoán vị $\sigma$ trên ​$A$. Ta gọi hai phần tử $i$​ và $j$​ tạo thành \textbf{nghịch thế} (inversion) nếu $i < j$​ và $\sigma(i) > \sigma(j)$.

    Đặt $\sigma = \{k_1, k_2, \cdots, k_n\}$​ là một hoán vị của $A$​. Ta ký hiệu \[ P\{k_1, k_2, \cdots, k_n\} \] là số lượng nghịch thế của $\sigma$​ và đặt \[ (-1)^{P\{k_1, k_2, \cdots, k_n\}} = \sign \{k_1, k_2, \cdots, k_n\}. \]
\end{definition}

\begin{example}
    Với $n=4$​, $A = \{1, 2, 3, 4\}$​. Xét hoán vị $\sigma = \{4, 2, 1, 3\}$.

    Ta nhận thấy các cặp nghịch thế $(4, 2),\ (4, 1),\ (4, 3),\ (2, 1)$​ gồm 4 cặp nghịch thế. Vậy $P\{4, 2, 1, 3\} = 4$ và $\sign \{4, 2, 1, 3\}=(-1)^4=1$​.
\end{example}

\begin{definition}[Định thức]
    Khi đó định thức của ma trận $\bm{A} = \begin{pmatrix}a_{11} & a_{12} & \cdots & a_{1n} \\ a_{21} & a_{22} & \cdots & a_{2n} \\ \cdots & \cdots & \cdots & \cdots \\ a_{n1} & a_{n2} & \cdots & a_{nn}\end{pmatrix}$ được định nghĩa là:
    \begin{equation}
        \det(\bm{A})=\displaystyle{\sum_{(i_1, i_2, \cdots, i_n)} a_{1, i_1} \cdot a_{2, i_2} \cdot a_{n, i_n} \cdot \sign\{i_1, i_2, \cdots, i_n\}}
    \end{equation}
    với mọi hoán vị $(i_1, i_2, \cdots, i_n)$ của $(1, 2, \cdots, n)$. Như vậy có $n!$​ phần tử cho tổng trên.
\end{definition}


\begin{example}
    Tính định thức ma trận $\bm{A}=\begin{pmatrix}1 & 2 & 3 \\ 4 & 5 & 6 \\ 7 & 8 & 9\end{pmatrix}$​.

    Xét hoán vị $\sigma_1 = \{1, 2, 3\}$​. Khi đó $P\{1, 2, 3\}=0$​, $a_{11} \cdot a_{22} \cdot a_{33} \cdot (-1)^0 = 1 \cdot 5 \cdot 9 \cdot 1 = 45$​.

    Xét hoán vị $\sigma_2 = \{1, 3, 2\}$​. Khi đó $P\{1, 3, 2\} = 1$​, $a_{11} \cdot a_{23} \cdot a_{32} \cdot (-1)^1 = 1 \cdot 6 \cdot 8 \cdot (-1) = -48$.

    Xét hoán vị $\sigma_3 = \{2, 1, 3\}$​. Khi đó $P\{2,1,3\}=1$​, $a_{12} \cdot a_{21} \cdot a_{33} \cdot (-1)^1 = 2 \cdot 4 \cdot 9 \cdot (-1) = -72$.

    Xét hoán vị $\sigma_4=\{2,3,1\}$. Khi đó $P\{2, 3, 1\} = 2$​, $a_{12} \cdot a_{23} \cdot a_{31} \cdot (-1)^2 = 2 \cdot 6 \cdot 7 \cdot 1 = 84$​.

    Xét hoán vị $\sigma_5=\{3, 1, 2\}$. Khi đó $P\{3, 1, 2\} = 2$​, $a_{13} \cdot a_{21} \cdot a_{32} \cdot (-1)^2 = 3 \cdot 4 \cdot 8 \cdot 1 = 96$​.

    Xét hoán vị $\sigma_6=\{3, 2, 1\}$​. Khi đó $P\{3, 2, 1\}=3$​, $a_{13} \cdot a_{22} \cdot a_{31} \cdot (-1)^3 = 3 \cdot 5 \cdot 7 \cdot  (-1) = -105$​.

    Như vậy $\det(A)=45-48-72+84+96-105=0$​.
\end{example}

Định thức của ma trận còn được định nghĩa theo \textbf{đệ quy} như sau:

Với ma trận $1 \times 1$ là $\bm{A}=\begin{pmatrix}a_{11}\end{pmatrix}$ thì $\det(\bm{A})=a_{11}$.

Với ma trận $2 \times 2$ là $\bm{A} = \begin{pmatrix}a_{11} & a_{12} \\ a_{21} & a_{22}\end{pmatrix}$​ thì $\det(\bm{A})=a_{11}a_{22} - a_{21}a_{12}$.

Với ma trận $n \times n$, gọi $\bm{M}_{ij}$ là ma trận có được từ ma trận $\bm{A}$ khi bỏ đi hàng $i$​ và cột $j$​ của ma trận $\bm{A}$ và ký hiệu $A_{ij}=(-1)^{i+j} \det (\bm{M}_{ij})$. Khi đó:

\begin{theorem}[Định lý Laplace]
    Định lý Laplace cho phép ta khai triển định thức của ma trận cấp $n$ thành tổng các ma trận cấp $n-1$.

    Khai triển theo cột $j$​: \[ \det(\bm{A})=\displaystyle{\sum_{i=1}^na_{ij} A_{ij}} = a_{1j} A_{1j} + a_{2j} A_{2j} + \cdots + a_{nj} A_{nj},\ j = \overline{1, n}.\]

    Khai triển theo hàng $i$​: \[ \det(\bm{A})=\displaystyle{\sum_{j=1}^n a_{ij} A_{ij}} = a_{i1} A_{i1} + a_{i2} A_{i2} + \cdots + a_{in} A_{in},\ i = \overline{1, n}. \]​
\end{theorem}

\subsection*{Hạng của ma trận}

\begin{definition}[Hạng của ma trận]
    Cho ma trận $\mathbf{A}_{m \times n}$. \textbf{Hạng} của ma trận là cấp của ma trận con lớn nhất có định thức khác 0. Nghĩa là một ma trận vuông mà là ma trận con (lấy 1 phần của ma trận gốc) kích thước $r \times r$ mà có định thức khác 0, thì hạng của ma trận khi đó là $r$. Dễ thấy do là ma trận con, và vuông, nên $r \leqslant \min(m, n)$.
\end{definition}

\begin{example}
    Ma trận $\bm{A} = \begin{pmatrix}
        1 & 2 & 3 \\ 2 & 4 & 6 \\ 1 & 2 & 4
    \end{pmatrix}$ có định thức $\det(\bm{A}) = 0$. 
    
    Nhưng ma trận con của $\bm{A}$ là $\bm{B} = \begin{pmatrix}2 & 3 \\ 2 & 4\end{pmatrix}$ (lấy dòng 1 và 3, lấy cột 2 và 3) có định thức $\det(\bm{B}) = 2 \neq 0$, do đó $r = \text{rank}(\bm{A}) = 2$ ($\text{rank}(\bm{A})$ nghĩa là hạng của $\bm{A}$).
\end{example}

\section{Ma trận nghịch đảo}

Ma trận $\bm{A}^{-1}$​ được gọi là \textbf{ma trận nghịch đảo} của ma trận vuông $\bm{A}$ nếu $\bm{A}^{-1} \cdot \bm{A} = \bm{A} \cdot \bm{A}^{-1} = \bm{I}$​. Trong đó $\bm{I}$ là ma trận đơn vị cùng kích thước với $\bm{A}$.

\begin{equation}
    \bm{A}^{-1}=\frac{1}{\det(\bm{A})}[(A_{ij})_n]^T=\frac{1}{\det(\bm{A})}\begin{pmatrix} A_{11} & A_{21} & \cdots & A_{n1} \\ A_{12} & A_{22} & \cdots & A_{n2} \\ \cdots & \cdots & \cdots & \cdots \\ A_{1n} & A_{2n} & \cdots & A_{nn} \end{pmatrix}
\end{equation}
Trong đó, $A_{ij}$ cũng được định nghĩa tương tự như khi tính định thức bằng khai triển theo dòng hoặc cột. Gọi $\bm{M}_{ij}$ là ma trận có được từ ma trận $\bm{A}$ khi bỏ đi hàng $i$​ và cột $j$​ của ma trận $\bm{A}$ và ký hiệu $A_{ij}=(-1)^{i+j} \det (\bm{M}_{ij})$.

Như vậy, điều kiện cần và đủ để một ma trận có nghịch đảo là định thức khác 0.