\chapter{Quantum computing}

\section{Qubit và toán tử quantum}

Trên máy tính hiện nay, đơn vị xử lý cơ bản là bit (0 hoặc 1). Trong máy tính lượng tử, đơn vị tính toán là qubit (quantum bit).

\subsection*{Qubit}

Mỗi qubit $\lvert \psi \rangle$ được biểu diễn dưới dạng tổ hợp tuyến tính của cơ sở gồm $\lvert 0 \rangle = (1, 0)$ và $\lvert 1 \rangle = (0, 1)$. Khi đó qubit $\lvert \psi \rangle = \alpha \lvert 0 \rangle + \beta \lvert 1 \rangle$. Ở đây $\alpha, \beta \in \mathbb{C}$ (tập số phức).

Tích của $n$ qubit là các tổ hợp $\lvert 0, 0, \ldots, 0 \rangle$, $\lvert 0, 0, \ldots, 1 \rangle$, ..., $\lvert 1, 1, \ldots, 1 \rangle$. Ta cũng ký hiệu $\lvert 0 \rangle \otimes \lvert 1 \rangle = \lvert 01 \rangle$. 

\begin{example}
    Nếu $\lvert \psi \rangle = \alpha \lvert 0 \rangle + \beta \lvert 1 \rangle$ và $\lvert \Psi \rangle = \gamma \lvert 0 \rangle + \delta \lvert 1 \rangle$ thì
    \begin{equation*}
        \lvert \psi \rangle \otimes \lvert \Psi \rangle = (\alpha \lvert 0 \rangle + \beta \lvert 1 \rangle) \otimes (\gamma \lvert 0 \rangle + \delta \lvert 1 \rangle) = \alpha \gamma \lvert 00 \rangle + \alpha \delta \lvert 0 1 \rangle + \beta \gamma \lvert 10 \rangle + \beta \delta \lvert 11 \rangle
    \end{equation*}
\end{example}

Tiếp theo là \textbf{toán tử quantum} và tương ứng với nó là các cổng (gate) trên mạch.

Toán tử quantum tác động lên một qubit hoặc tích của nhiều qubit.

Qubit có dạng $\lvert \psi \rangle = a \lvert 0 \rangle + b \lvert 1 \rangle$. Ta có thể viết hệ số dưới dạng vector cột $\begin{pmatrix} a \\ b \end{pmatrix}$. Khi đó, toán tử quantum sẽ là một ma trận $2 \times 2$ biến đổi vector trên thành vector mới $\begin{pmatrix} c \\ d \end{pmatrix}$ tương ứng với qubit $\lvert \Psi \rangle = c \lvert 0 \rangle + d \lvert 1 \rangle$.

Nói cách khác, đặt toán tử quantum là ma trận $\mathcal{U} = \begin{pmatrix} c_{11} & c_{12} \\ c_{21} & c_{22} \end{pmatrix}$ thì ta có
\begin{equation*}
    \lvert \psi \rangle \to \lvert \Psi \rangle = \mathcal{U} \lvert \psi \rangle, \quad \begin{pmatrix} c_{11} & c_{12} \\ c_{21} & c_{22} \end{pmatrix} \cdot \begin{pmatrix} a \\ b \end{pmatrix} = \begin{pmatrix} c \\ d \end{pmatrix}
\end{equation*}

\subsection*{Các toán tử quantum thường gặp}

1. \textbf{Toán tử đồng nhất}. Toán tử đồng nhất identity giữ nguyên qubit đầu vào. Ma trận tương ứng là ma trận đơn vị $I = \begin{pmatrix} 1 & 0 \\ 0 & 1 \end{pmatrix}$.

2. \textbf{Toán tử NOT}. Toán tử NOT có ma trận tương ứng là $\text{NOT} = \begin{pmatrix} 0 & 1 \\ 1 & 0 \end{pmatrix}$. Khi đó $\text{NOT} \lvert \psi \rangle = b \lvert 0 \rangle + a \lvert 1 \rangle$ với $x \in \{ 0, 1 \}$.

Khi qubit là $\lvert 0 \rangle$ hoặc $\lvert 1 \rangle$, tác dụng của toán tử NOT là phép XOR nên ta có $\text{NOT} \lvert x \rangle = \lvert x \oplus 1 \rangle$.

3. \textbf{Toán tử Hadamard}. Đây là một toán tử đặc biệt và được quan tâm nhiều.

Ma trận của toán tử Hadamard là $H = \dfrac{1}{\sqrt{2}} \begin{pmatrix} 1 & 1 \\ 1 & - 1 \end{pmatrix}$. 

\begin{example}
    Xét qubit $\lvert \psi \rangle = a \lvert 0 \rangle + b \lvert 1 \rangle$, toán tử Hadamard tương ứng với phép nhân ma trận
    \begin{equation*}
        \dfrac{1}{\sqrt{2}} \begin{pmatrix} 1 & 1 \\ 1 & - 1 \end{pmatrix} \cdot \begin{pmatrix} a \\ b \end{pmatrix} = \begin{pmatrix} \dfrac{1}{\sqrt{2}} (a + b) \\ \dfrac{1}{\sqrt{2}} (a - b) \end{pmatrix}
    \end{equation*}

    Ta chuyển cột kết quả về lại dạng tổ hợp tuyến tính thì cổng Hadamard hoạt động trên qubit $\lvert \psi \rangle = a \lvert 0 \rangle + b \lvert 1 \rangle$ cho kết quả là
    \begin{equation*}
        H \lvert \psi \rangle = H(a \lvert 0 \rangle + b \lvert 1 \rangle) = \dfrac{1}{\sqrt{2}} (a + b) \lvert 0 \rangle + \dfrac{1}{\sqrt{2}} (a - b) \lvert 1 \rangle
    \end{equation*}
\end{example}

Nếu $\lvert \psi \rangle \equiv \lvert 0 \rangle$ thì tương đương với $a = 1, b = 0$. Ta có $H \lvert 0 \rangle = \dfrac{\lvert 0 \rangle + \lvert 1 \rangle}{\sqrt{2}}$.

Nếu $\lvert \psi \rangle \equiv \lvert 1 \rangle$ thì tương đương với $a = 0, b = 1$. Ta có $H \lvert 1 \rangle = \dfrac{\lvert 0 \rangle - \lvert 1 \rangle}{\sqrt{2}}$.

Tổng quát ta nhận thấy, với $x \in \{ 0, 1 \}$ thì $H \lvert x \rangle = \dfrac{\lvert 0 \rangle + (-1)^x \lvert 1 \rangle}{\sqrt{2}}$.

Ta thấy rằng toán tử ngược của toán tử Hadamard là chính nó.

4. \textbf{Toán tử control}. Đây là toán tử thường được dùng nhất khi tính toán trên tích của nhiều qubit.

Như đã xem xét ở trên, tích của $n$ qubit sẽ có $2^n$ phần tử tương ứng các bộ $\lvert 0, 0, \ldots, 0, 0 \rangle$, $\lvert 0, 0, \ldots, 0, 1 \rangle$, ... Do đó toán tử control sẽ là ma trận kích thước $2^n \times 2^n$.

Gọi $\mathcal{U} = \begin{pmatrix} c_{11} & c_{12} \\ c_{21} & c_{22} \end{pmatrix}$ là toán tử tác động lên một qubit (ví dụ như 3 toán tử đã đề cập). Xét hai qubit là $\lvert x \rangle = a \lvert 0 \rangle + b \lvert 1 \rangle$ và $\lvert y \rangle = c \lvert 0 \rangle + d \lvert 1 \rangle$. Từ phía trên
\begin{equation*}
    \lvert x \rangle \otimes \lvert y \rangle = ac \lvert 00 \rangle + ad \lvert 01 \rangle + bc \lvert 10 \rangle + bd \lvert 11 \rangle
\end{equation*}

Khi đó toán tử control có dạng ma trận là
\begin{equation*}
    c \mathcal{U} = \begin{pmatrix} 1 & 0 & 0 & 0 \\ 0 & 1 & 0 & 0 \\ 0 & 0 & c_{11} & c_{12} \\ 0 & 0 & c_{21} & c_{22} \end{pmatrix}
\end{equation*}

Hay viết dưới dạng ma trận khối là $c \mathcal{U} = \begin{pmatrix} I & \mathcal{O} \\ \mathcal{O} & \mathcal{U} \end{pmatrix}$.

Ta cũng viết tích $\lvert x \rangle \otimes \lvert y \rangle$ dưới dạng vector cột (4 phần tử). Khi đó
\begin{equation*}
    \mathcal{U} (\lvert x \rangle \otimes \lvert y \rangle) = \begin{pmatrix} 1 & 0 & 0 & 0 \\ 0 & 1 & 0 & 0 \\ 0 & 0 & c_{11} & c_{12} \\ 0 & 0 & c_{21} & c_{22} \end{pmatrix} \cdot \begin{pmatrix} ac \\ ad \\ bc \\ bd \end{pmatrix} = \begin{pmatrix} ac \\ ad \\ c_{11} \cdot bc + c_{12} \cdot bd \\ c_{21} \cdot bc + c_{22} \cdot bd \end{pmatrix}
\end{equation*}

Hai phần tử đầu của vector kết quả không thay đổi, còn phần sau có "một phần" là $\mathcal{U} \lvert y \rangle$. Khi viết lại kết quả dưới dạng qubit thì
\begin{equation*}
    ac \lvert 00 \rangle + ad \lvert 01 \rangle + (c_{11} \cdot bc + c_{12} \cdot bd) \lvert 10 \rangle + (c_{21} \cdot bc + c_{22} \cdot bd) \lvert 11 \rangle
\end{equation*}

Ta có một số nhận xét sau đây.

Nếu $\lvert x \rangle \equiv \lvert 0 \rangle$, tức là $a = 1, b = 0$ thì tích trên tương ứng với $c \lvert 00 \rangle + d \lvert 01 \rangle + 0 \lvert 10 \rangle + 0 \lvert 11 \rangle = \lvert 0 \rangle \otimes (c \lvert 0 \rangle + d \lvert 1 \rangle) = \lvert x \rangle \otimes \lvert y \rangle$.

Nếu $\lvert x \rangle \equiv \lvert 1 \rangle$, tức là $a = 0, b = 1$ thì tích trên tương ứng với $0 \lvert 00 \rangle + 0 \lvert 01 \rangle + (c_{11} c + c_{12} d) \lvert 10 \rangle + (c_{21} c + c_{22} d) \lvert 11 \rangle = \lvert 1 \rangle \otimes ((c_{11} c + c_{12} d) \lvert 0 \rangle + (c_{21} c + c_{22} d) \lvert 1 \rangle) = \lvert 1 \rangle \otimes \mathcal{U} \lvert y \rangle = \lvert x \rangle \otimes \mathcal{U} \lvert y \rangle$.

Tổng kết lại, với $x \in \{ 0, 1\}$ thì

\begin{itemize}
    \item nếu $x = 0$ thì $\lvert x \rangle \otimes \lvert y \rangle \to \lvert x \rangle \otimes \lvert y \rangle$.
    \item nếu $x = 1$ thì $\lvert x \rangle \otimes \lvert y \rangle \to \lvert x \rangle \otimes \mathcal{U} \lvert y \rangle$.
\end{itemize}

Tùy vào $x$ là 0 hay 1 mà toán tử quantum $\mathcal{U}$ sẽ bị bỏ qua hoặc xem xét. Ở đây qubit $\lvert x \rangle$ đóng vai trò điều khiển nên đây được gọi là toán tử control.

5. \textbf{Toán tử control CNOT (Control NOT)}. Toán tử quantum CNOT có ma trận là
\begin{equation*}
    \begin{pmatrix} 1 & 0 & 0 & 0 \\ 0 & 1 & 0 & 0 \\ 0 & 0 & 0 & 1 \\ 0 & 0 & 1 & 0 \end{pmatrix} = \begin{pmatrix} I & \mathcal{O} \\ \mathcal{O} & \text{NOT} \end{pmatrix}
\end{equation*}

Qubit $\lvert x \rangle$ với $x \in \{ 0, 1 \}$ đóng vai trò control cho qubit $\lvert y \rangle$. Khi $x \equiv 0$ thì $y$ giữ nguyên, hay $\lvert y \oplus 0 \rangle = \lvert y \oplus x \rangle$. Khi $x \equiv 1$ thì áp dụng cổng NOT bên trên, khi đó $y$ biến đổi thành $y \oplus 1 = y \oplus x$.

\subsection*{Một ví dụ về qubit trong NSUCRYPTO 2022}

Đố vui: nếu ta có qubit là $\alpha \lvert 0 \rangle + \beta \lvert 1 \rangle$, hãy xây dựng mạch logic để biến đổi qubit trên thành $\alpha \lvert 000 \rangle + \beta \lvert 111 \rangle$.

Giải: sử dụng toán tử Hadamard và NOT. Ta có \[ NOT(\alpha \lvert 0 \rangle + \beta \lvert 1 \rangle) = \beta \lvert 0 \rangle + \alpha \lvert 1 \rangle\] và \[\mathcal{H} (\alpha \lvert 0 \rangle + \beta \lvert 1 \rangle) = \dfrac{1}{\sqrt{2}} (\alpha + \beta) \lvert 0 \rangle + \dfrac{1}{\sqrt{2}} (\alpha - \beta) \lvert 1 \rangle \]

Toán tử $CNOT$ được biểu diễn bởi ma trận \[ CNOT = \begin{pmatrix}
    1 & 0 & 0 & 0 \\ 0 & 1 & 0 & 0 \\ 0 & 0 & 0 & 1 \\ 0 & 0 & 1 & 0
\end{pmatrix}\]

Giả sử ta có hai qubit là $\lvert \alpha \rangle = x \lvert 0 \rangle + y \lvert 1 \rangle$ và $\lvert \beta \rangle = z \lvert 0 \rangle + t \lvert 1 \rangle$.

Khi đó $\lvert \alpha \rangle \otimes \lvert \beta \rangle = xz \lvert 00 \rangle + xt \lvert 01 \rangle + yz \lvert 10 \rangle + yt \lvert 11 \rangle$.

Qua toán tử $CNOT$ ta có

\begin{align*}
    CNOT (\lvert \alpha \rangle \otimes \lvert \beta \rangle) 
    = & \begin{pmatrix}
    1 & 0 & 0 & 0 \\ 0 & 1 & 0 & 0 \\ 0 & 0 & 0 & 1 \\ 0 & 0 & 1 & 0
    \end{pmatrix} \begin{pmatrix}
    xz \\ xt \\ yz \\ yt
    \end{pmatrix}
    = \begin{pmatrix}
    xz \\ xt \\ yt \\ yz
    \end{pmatrix} \\
    = & xz \lvert 00 \rangle + xt \lvert 01 \rangle + yt \lvert 10 \rangle + yz \lvert 11 \rangle
\end{align*}

Trường hợp $\lvert \alpha \rangle = \lvert 0 \rangle$ thì $x = 1, y = 0$. Khi đó ta có tương đương 
\begin{equation*}
    CNOT (\lvert 0 \rangle \otimes \lvert \beta \rangle) = z \lvert 00 \rangle + t \lvert 01 \rangle = \lvert 0 \rangle \otimes (z \lvert 0 \rangle + t \lvert 1 \rangle) = \lvert 0 \rangle \otimes \lvert \beta \rangle.
\end{equation*}

Trường hợp $\lvert \alpha \rangle = \lvert 1 \rangle$ thì $x = 0, y = 1$. Khi đó ta có tương đương 
\begin{equation*}
    CNOT (\lvert 1 \rangle) \otimes \lvert \beta \rangle) = t \lvert 10 \rangle + z \lvert 11 \rangle = \lvert 1 \rangle \otimes (t \lvert 0 \rangle + z \lvert 1 \rangle) = \lvert 1 \rangle \otimes NOT(\lvert \beta \rangle)
\end{equation*}

Nói cách khác, nếu $x \in \{ 0, 1 \}$ thì 

\begin{equation*}
    CNOT (\lvert x \rangle \otimes \lvert \beta \rangle) = 
    \begin{cases}
        \lvert x \rangle \otimes \lvert \beta \rangle, & \text{nếu } x = 0 \\ 
        \lvert x \rangle \otimes NOT(\lvert \beta \rangle), & \text{nếu } x = 1
    \end{cases}
\end{equation*}

Do đó các toán tử có ma trận $\begin{pmatrix}
    I_n & \mathcal{O} \\ \mathcal{O} & \mathcal{U}
\end{pmatrix}$ được gọi là toán tử kiểm soát (controlled).

Một trường hợp riêng nữa là khi $\beta = 0$ hoặc $\beta = 1$. Khi đó, với toán tử $NOT$ bên trên ta suy ra 
\begin{equation*}
    CNOT (\lvert x \rangle \otimes \lvert y \rangle) = \lvert x \rangle \otimes \lvert x \oplus y \rangle
\end{equation*}
