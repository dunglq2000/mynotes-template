\chapter*{Chuyên đề 1. Lý thuyết Galois}
\addcontentsline{toc}{chapter}{Chuyên đề 1. Lý thuyết Galois}

Tài liệu tham khảo trong phần này lấy từ sách \cite{Judson2012} và trang youtube\footnote{\url{https://youtu.be/Buv4Y74_z7I?si=sQ0lOodnuLR0Yb00}}.

\section{Extension Fields}

\begin{definition}[Mở rộng trường (Extension Field)]
    Trường $E$ được gọi là \textbf{mở rộng trường} (hay \textbf{extension field}) của trường $F$ nếu $F$ là trường con của $E$. Khi đó $F$ được gọi là \textbf{trường cơ sở} (hay \textbf{base field}) và ký hiệu $F \subset E$.
\end{definition}

\begin{example}
    Trường nào nhỏ nhất chứa $\QQ$ và $\sqrt{2}$?

    Đáp án là trường $\QQ(\sqrt{2}) = \{ a + b \sqrt{2} : a, b \in \QQ \}$.

    Việc chứng minh $\QQ(\sqrt{2})$ là trường khá đơn giản, phần tử nghịch đảo đối với phép nhân của $a + b \sqrt{2}$ là
    \begin{equation*}
        \frac{a}{a^2 - 2 b^2} + \frac{-2b}{a^2 - 2 b^2} \sqrt{2}
    \end{equation*}
\end{example}

\begin{example}
    Trường nào nhỏ nhất chứa $\QQ$ và $i$ (ở đây $i$ là đơn vị ảo, $i^2 = -1$)?

    Đáp án là trường $\QQ(i) = \{ a + b i : a, b \in \QQ \}$. Tương tự, ở đây phần tử nghịch đảo đối với phép nhân của $a + b i$ là
    \begin{equation*}
        \frac{a}{a^2 + b^2} + \frac{-b}{a^2 + b^2} i
    \end{equation*}
\end{example}

Ở đây $\QQ(\sqrt{2})$ và $\QQ(i)$ đều là mở rộng của $\QQ$ và đều là tập con của $\CC$. Tuy nhiên hai trường này không phải tập con của nhau.

Như vậy, bằng việc mở rộng $\QQ$ với $\sqrt{2}$ ta có trường $\QQ(\sqrt{2})$.

Tương tự, bằng việc mở rộng $\QQ$ với $i$ ta có trường $\QQ(i)$.

Vậy trường nào chứa $\QQ$, $\sqrt{2}$ và $i$?

\begin{example}
    Trường chứa cả $\QQ$, $\sqrt{2}$ và $i$ là tập hợp
    \begin{equation*}
        \QQ(\sqrt{2}, i) = \{ a + b\sqrt{2} + c i + d \sqrt{2} i : a, b, c, d \in \QQ \}
    \end{equation*}
\end{example}

Trường trên có thể suy ra từ logic sau. Ta đã có $\QQ(i)$ chứa $\QQ$ và $i$. Ta muốn thêm $\sqrt{2}$ vào trường $\QQ(i)$ nên ta sẽ muốn mở rộng $\QQ(i)$ lên $\QQ(i)(\sqrt{2})$.

Khi đó $\QQ(i)(\sqrt{2})$ tương tự sẽ có dạng
\begin{equation*}
    \QQ(i)(\sqrt{2}) = \{ \alpha + \beta \sqrt{2} : \alpha, \beta \in \QQ(i) \}
\end{equation*}

Nói cách khác $\alpha = a + b i$ và $\beta = c + d i$, $a, b, c, d \in \QQ$, nên ta có
\begin{equation*}
    \alpha + \beta \sqrt{2} = a + b i + c \sqrt{2} + d \sqrt{2} i, \quad a, b, c, d \in \QQ
\end{equation*}

Khi đó ta viết
\begin{equation*}
    \QQ(i)(\sqrt{2}) = \QQ(\sqrt{2}, i) = \{ a + b\sqrt{2} + c i + d \sqrt{2} i : a, b, c, d \in \QQ \}
\end{equation*}

\begin{remark}
    \begin{enumerate}
        \item $\QQ(\sqrt{2})$ là trường con của $\RR$ nhưng $\QQ(i)$ không phải;
        \item $\QQ(i)$ nhỏ hơn $\CC$ rất nhiều (không chứa $\sqrt{2}$);
        \item $\QQ(\sqrt{2})$ chứa tất cả nghiệm của đa thức $f(x) = x^2 - 2$ trên $\QQ$. Do đó $\QQ(\sqrt{2})$ được gọi là \textbf{trường phân rã} của đa thức $f(x)$.
    \end{enumerate}
\end{remark}

\begin{definition}[Trường phân rã (Splitting Field)]
    Xét trường $F$ và đa thức khác hằng $p(x) = a_n x^n + a_{n-1} x^{n-1} + \ldots + a_1 x + a_0$ trên $F[x]$ ($a_i \in F$ với mọi $i = 1, 2, \ldots$).

    Trường mở rộng $E$ của trường $F$ được gọi là \textbf{trường phân rã} (hay \textbf{splitting field}) của $p(x)$ nếu tồn tại các phần tử $\alpha_1, \alpha_2, \ldots, \alpha_n$ thuộc $E$ sao cho $E = F(\alpha_1, \alpha_2, \ldots, \alpha_n)$ và
    \begin{equation*}
        p(x) = (x - \alpha_1) (x - \alpha_2) \cdots (x - \alpha_n)
    \end{equation*}
\end{definition}

Khi đó ta nói đa thức $p(x) \in F[x]$ phân rã (split) trong $E$ nếu nó phân tích thành các nhân tử bậc nhất (tuyến tính) trong $E[x]$.

Nói nôm na, nếu đa thức có hệ số trong một trường $F$ nào đó (tức thuộc $F[x]$) thì các nghiệm của nó nằm trong một trường lớn hơn chứa $F$.

\begin{example}
    Đa thức $f(x) = x^2 - 2$ trên $\QQ[x]$ không có nghiệm trên $\QQ$. Tuy nhiên $\QQ(\sqrt{2})$ là trường chứa $\QQ$ và các nghiệm của $f(x)$ là $\pm \sqrt{2}$. Vì vậy $\QQ(\sqrt{2})$ là trường phân rã của $f(x)$.
\end{example}

\begin{example}
    Đa thức $g(x) = x^2 + i$ trên $\QQ[x]$ không có nghiệm trên $\QQ$. Tuy nhiên $g(x)$ có hai nghiệm là $\pm i$ và $\QQ(i)$ chứa cả $\QQ$ và $\pm i$ nên $\QQ(i)$ là một trường phân rã của $g(x)$.
\end{example}

\begin{example}
    Xét trường
    \begin{equation*}
        F = \QQ(\sqrt{2}) = \{ a + b  \sqrt{2}: a, b \in \QQ\}
    \end{equation*}

    Ta gọi $E = \QQ(\sqrt{2} + \sqrt{3})$ là trường nhỏ nhất chứa cả $\QQ$ và $\sqrt{2} + \sqrt{3}$. Ta thấy rằng cả $E$ và $F$ đều là mở rộng của $\QQ$.

    Ta sẽ chứng minh $E$ là một mở rộng trường của $F$.

    \begin{proof}
        Do $\sqrt{2} + \sqrt{3} \in E$ nên nghịch đảo của nó cũng nằm trong $E$, nghĩa là $\dfrac{1}{\sqrt{2} + \sqrt{3}} \in E$. Suy ra $\sqrt{3} - \sqrt{2} \in E$.

        Trong trường, mọi tổ hợp tuyến tính của hai phần tử bất kỳ luôn cho kết quả là phần tử trong trường. Do đó $\dfrac{1}{2} \cdot (\sqrt{2} + \sqrt{3}) - \dfrac{1}{2} \cdot (\sqrt{3} - \sqrt{2}) = \sqrt{2} \in E$.

        Tương tự $\sqrt{3} \in E$. Do $E$ chứa $\QQ$ và chứa $\sqrt{2}$ nên $E$ chứa tất cả phần tử dạng $a + b \sqrt{2}$ với $a, b \in \QQ$ (tổ hợp tuyến tính của 1 và $\sqrt{2}$) nên $E$ là một mở rộng của $F$.
    \end{proof}
\end{example}

%\begin{example}
%    Ta xét trường
%    \begin{equation*}
%        \QQ(\sqrt{2}; \sqrt{3}) = \{ a + b \sqrt{2} + c \sqrt{3} + d \sqrt{6} : a, b, c, d \in \QQ \}
%    \end{equation*}

%    Ta xét trường $\QQ(\sqrt{2} + \sqrt{3})$ là trường nhỏ nhất chứa cả $\QQ$ và $\sqrt{2} + \sqrt{3}$.

%    Ta sẽ chứng minh $\QQ(\sqrt{2}; \sqrt{3}) \equiv \QQ(\sqrt{2} + \sqrt{3})$.

%    \begin{proof}
%        1. Ta chứng minh $\QQ(\sqrt{2}; \sqrt{3}) \subset \QQ(\sqrt{2} + \sqrt{3})$.

%        Do $\sqrt{2} + \sqrt{3}$ nên nghịch đảo $\dfrac{1}{\sqrt{2} + \sqrt{3}} = \sqrt{3} - \sqrt{2} \in \QQ(\sqrt{2} + \sqrt{3})$.

%        Khi đó, $\dfrac{1}{2} \cdot (\sqrt{2} + \sqrt{3}) \pm \dfrac{1}{2} \cdot (\sqrt{3} - \sqrt{2}) \in \QQ(\sqrt{2} + \sqrt{3})$.

%        Vế trái sẽ bằng $\sqrt{2}$ hoặc $\sqrt{3}$. Như vậy $\sqrt{2}, \sqrt{3} \in \QQ(\sqrt{2} + \sqrt{3})$.

%        Phép nhân trên trường cho ta $\sqrt{2} \cdot \sqrt{3} = \sqrt{6} \in \QQ(\sqrt{2} + \sqrt{3})$.

%        Như vậy với mọi $a, b, c, d \in \QQ$ ta có $a + b \sqrt{2} + c \sqrt{3} + d \sqrt{6} \in \QQ(\sqrt{2} + \sqrt{3})$. Điều này tương đương với $\QQ(\sqrt{2}; \sqrt{3}) \subset \QQ(\sqrt{2} + \sqrt{3})$.

%        2. Ta chứng minh $\QQ(\sqrt{2} + \sqrt{3}) \subset \QQ(\sqrt{2}; \sqrt{3})$.

%        Theo định nghĩa $\QQ(\sqrt{2}; \sqrt{3})$ ta có $\sqrt{2}, \sqrt{3} \in \QQ(\sqrt{2}; \sqrt{3})$.

%        Từ đây suy ra $\sqrt{2} + \sqrt{3} \in \QQ(\sqrt{2}; \sqrt{3})$, mà $\QQ \subset \QQ(\sqrt{2}; \sqrt{3})$ nên cuối cùng $\QQ(\sqrt{2} + \sqrt{3}) \subset \QQ(\sqrt{2}; \sqrt{3})$.

%        Từ hai điều trên ta có $\QQ(\sqrt{2} + \sqrt{3}) \equiv \QQ(\sqrt{2}; \sqrt{3})$.
%    \end{proof}
%\end{example}