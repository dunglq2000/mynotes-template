\documentclass{article}

\usepackage{amsmath,amssymb}

\title{Problem 11. Ant cipher}
\author{Dung Le Quoc}
\date{\today}

\begin{document}
\maketitle

\section{Problem}

The cipher must be represented by the equation CNF=True. In Sam's CNF, $x_1$ and $x_2$ correspond to the plaintext, $x_9$ and $x_{10}$ correspond to the ciphertext, while the remaining 6 variables are auxiliary. The equation is as follow:
\begin{align*}
    (x_1 \lor x_2 \lor x_9) \land (\lnot x_1 \lor \lnot x_2 \lor \lnot x_9) \land (\lnot x_1 \lor x_2 \lnot x_9) \land (x_1 \lor \lnot x_2 \lor x_9) \land \\
    (x_1 \lor x_2 \lor x_3) \land (\lnot x_9 \lor \lnot x_{10} \lor \lnot x_3) \land (x_1 \lor \lnot x_2 \lor x_4) \land (\lnot x_9 \lor x_{10} \lor \lnot x_4) \land \\
    (\lnot x_1 \lor x_2 \lor x_5) \land (x_9 \lor \lnot x_{10} \lor \lnot x_5) \land (\lnot x_1 \lor \lnot x_2 \lor x_6) \land (x_9 \lor x_{10} \lor \lnot x_6) \land \\
    (x_1 \lor x_2 \lor x_3 \lor x_4 \lor \lnot x_7) \land (x_2 \lor x_3 \lor x_4 \lor \lnot x_7 \lor \lnot x_8) = True
\end{align*}

\section{Solution}

By examing truth table of equation above, I get some notice. Suppose that I write vectors by order $(x_1, x_2, \ldots, x_{10})$. 

According to the problem, $(x_1, x_2)$ is encrypted to $(x_9, x_{10})$ where the value in truth table is True.

This means that, where $f(x_1, x_2, \ldots, x_9, x_{10}) = 1$ with $f$ is boolean function above, then $(x_1, x_2)$ is encrypted to $(x_9, x_{10})$.

From truth table, I see that $(0, 0)$ is encrypted to $(1, 0)$, $(0, 1)$ is encrypted to $(1, 1)$, $(1, 0)$ is encrypted to $(0, 0)$, $(1, 1)$ is encrypted to $(0 ,1)$.

As a result, we can ignore all variables $x_3, x_4, x_5, x_6, x_7, x_8$, because they do not affect how we decrypt the ciphertext. This is because the encryption from $(x_1, x_2)$ to $(x_9, x_{10})$ is bijection.

In fact, we only need to consider truth table of 4 variables, where 
\begin{align*}
    f(0, 0, 1, 0) = 1, \\
    f(0, 1, 1, 1) = 1, \\
    f(1, 0, 0, 0) = 1, \\
    f(1, 1, 0, 1) = 1
\end{align*}

Full truth table is written in table \ref{func}.

\begin{table}[htb]
    \centering
    \begin{tabular}{|c|c|c|c|c|}
        \hline
        $x_1$ & $x_2$ & $x_9$ & $x_{10}$ & $f$ \\
        \hline
        0 & 0 & 0 & 0 & 0 \\
        \hline
        0 & 0 & 0 & 1 & 0 \\
        \hline
        0 & 0 & 1 & 0 & 1 \\
        \hline
        0 & 0 & 1 & 1 & 0 \\
        \hline
        0 & 1 & 0 & 0 & 0 \\
        \hline
        0 & 1 & 0 & 1 & 0 \\
        \hline
        0 & 1 & 1 & 0 & 0 \\
        \hline
        0 & 1 & 1 & 1 & 1 \\
        \hline
        1 & 0 & 0 & 0 & 1 \\
        \hline
        1 & 0 & 0 & 1 & 0 \\
        \hline
        1 & 0 & 1 & 0 & 0 \\
        \hline
        1 & 0 & 1 & 1 & 0 \\
        \hline
        1 & 1 & 0 & 0 & 0 \\
        \hline
        1 & 1 & 0 & 1 & 1 \\
        \hline
        1 & 1 & 1 & 0 & 0 \\
        \hline
        1 & 1 & 1 & 1 & 0 \\
        \hline
    \end{tabular}
    \caption{Boolean function $f$}
    \label{func}
\end{table}

By method Karnaugh map, I convert this truth table to CNF and receive the following equation

\begin{align*}
    f(x_1, x_2, x_9, x_{10}) = & (\lnot x_1 \lor \lnot x_9) \land (x_1 \lor x_9) \land \\ 
    & (\lnot x_1 \lor \lnot x_2 \lor x_{10}) \land (x_1 \lor x_2 \lor \lnot x_{10}) \land \\
    & (\lnot x_1 \lor x_2 \lor \lnot x_{10}) \land (x_1 \lor \lnot x_2 \lor x_{10})
\end{align*}

This CNF has four variables and 16 literals.

\end{document}